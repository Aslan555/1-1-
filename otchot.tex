%% -*- coding: utf-8 -*-
\documentclass[12pt,a4paper]{scrartcl} 
\usepackage[utf8]{inputenc}
\usepackage[english,russian]{babel}
\usepackage{indentfirst}
\usepackage{misccorr}
\usepackage{graphicx}
\usepackage{amsmath}
\usepackage{cmap}
\begin{document}
\section{Введение}
\label{sec:intro}

% Что должно быть во введении
\begin{enumerate}
 \item Написать приложение для вычисления корней квадратного уравнения (всех возможных вариантов и комплексности корней).
 \item Пример кода, решающего данную задачу
 \item Скриншот программы
\end{enumerate}

Пример приведен в пункте ~\ref{sec:exp} на стр.~\pageref{sec:exp}.

\section{Ход работы}
\label{sec:exp}

\subsection{Код приложения}
\label{sec:exp:code}
\begin{verbatim}
#include <iostream>
#include <cmath>
using namespace std;
int main ()
{
  double a;
  double b;
  double c;
  double x, s, d;
  cout << "введите а: ";
  cin >> a;
  cout << "введите b: ";
  cin >> b;
  cout << "введите c: ";
  cin >> c;
  if ((b * b - 4 * a * c) >= 0)
    {
      x = (-1 * b + sqrt (b * b - 4 * a * c)) / (2 * a);
      cout << "первый корень равен:" << x << endl;
      x = (-1 * b - sqrt (b * b - 4 * a * c)) / (2 * a);
      cout << "второй корень равен:" << x << endl;
    }
  else
  {
      s = -1*b/(2*a);
      d = sqrt (b * b - 4 * a * c) / (2 * a);
      cout << "x"<< s <<"+-i"<< d;
  }
  return 0;
    
}
\end{verbatim}

\subsection{Примеры формул}
\label{sec:mathexample}

Решение квадратного уравнения \(ax^2+bx+c=0\):
\begin{equation}\label{eq:solv}
 x_{1,2}=\frac{-b\pm\sqrt{b^2-4ac}}{2a}
\end{equation}

Решение через комплексные числа~\eqref{eq:solv}.
\begin{equation}\label{eq:solv}
 x=\frac{-b\pm{i}\sqrt{b^2-4ac}}{2a}
\end{equation}

\section{Код после выполнения программы}
\label{sec:picexample}
\begin{figure}[h]
	
	\includegraphics[width=0.6\textwidth]{код.jpg}
\end{figure}

\section{Пример библиографических ссылок}

Для написания «программы» необходимо 
изучить~\cite{Marius}, для использования \LaTeX{} лучше
почитать~\cite{Lvovsky-2003}, а для работы с Git~\cite{Git}.

\begin{thebibliography}{9}
\bibitem{Marius}Мариус Бансила. Решение задач на современном С++ \newblock --- Москва: Изд. ДМК, 2019 г. 295~с.
\bibitem{Lvovsky-2003}Львовский С.М. Набор и верстка в системе \LaTeX{}. \newblock --- 3-е издание, исправленное и дополненное, 2003 г.
\bibitem{Git} Скоттом Чаконом, Беном Штраубом Pro Git \newblock ---2-е издание 2014г.
\end{thebibliography}

\end{document}